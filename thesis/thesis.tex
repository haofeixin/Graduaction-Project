
%%%%%%%%%%%%%%%%%%%%%%%%%%%%%%%%%%%%%%%%%%%%%%%%%%%%%%%%%%%%%%%%%%%%%%
% njuthesis 示例模板 v1.4.2 2024-11-08
% https://github.com/nju-lug/NJUThesis
%
% 贡献者
% Yu XIONG @atxy-blip   Yichen ZHAO @FengChendian
% Song GAO @myandeg     Chang MA @glatavento
% Yilun SUN @HermitSun  Yinfeng LIN @linyinfeng
% Yukai Chou @Muzimuzhi
%
% 许可证
% LaTeX Project Public License(版本 1.3c 或更高)
%%%%%%%%%%%%%%%%%%%%%%%%%%%%%%%%%%%%%%%%%%%%%%%%%%%%%%%%%%%%%%%%%%%%%%

%---------------------------------------------------------------------
% 一些提升使用体验的小技巧:
%   1. 请务必使用 UTF-8 编码编写和保存本文档
%   2. 请务必使用 XeLaTeX 或 LuaLaTeX 引擎进行编译
%   3. 不保证接口稳定,写作前一定要留意版本号
%   4. 以百分号(%)开头的内容为注释,可以随意删除
%---------------------------------------------------------------------

%---------------------------------------------------------------------
% 请先阅读使用手册:
% http://mirrors.ctan.org/macros/unicodetex/latex/njuthesis/njuthesis.pdf
%---------------------------------------------------------------------

\documentclass[
    % 模板选项(注意右端逗号):
    %
    type = bachelor, % 文档类型,默认为本科生
    % degree = academic|professional,        % 学位类型,默认为学术型
    %
    % nl-cover,   % 是否需要国家图书馆封面,默认关闭
    % decl-page,  % 是否需要诚信承诺书或原创性声明,默认关闭
    %
    %   页面模式,详见手册说明
    % draft,                  % 开启草稿模式
    % anonymous,              % 开启盲审模式
    % minimal,                % 开启最小化模式
    %
    %   单双面模式,默认为适合印刷的双面模式
    oneside,                % 单面模式,无空白页
    % twoside,                % 双面模式,每一章从奇数页开始
    %
    %   字体设置,不填写则自动调用系统预装字体,详见手册
    % fontset = win|mac|macoffice|fandol|none,
  ]{njuthesis}

% 模板选项设置,包括个人信息、外观样式等
% 较为冗长且一般不需要反复修改,我们把它放在单独的文件里
\input{njuthesis-setup.def}

% 自行载入所需宏包
\usepackage{subcaption} % 嵌套小幅图像,比 subfig 和 subfigure 更新更好
\usepackage{graphicx}
\usepackage{booktabs}   % for three-line table
\usepackage{caption}    % for adjusting table captions
\usepackage{booktabs}
\usepackage{tabularx}
\usepackage{threeparttable}
% \usepackage{siunitx} % 标准单位符号
% \usepackage{physics} % 物理百宝箱
% \usepackage[version=4]{mhchem} % 绘制分子式
% \usepackage{listings} % 展示代码
% \usepackage{algorithm,algorithmic} % 展示算法伪代码

% 在导言区随意定制所需命令
% \DeclareMathOperator{\spn}{span}
% \NewDocumentCommand\mathbi{m}{\textbf{\em #1}}

% 开始编写论文
\begin{document}

%---------------------------------------------------------------------
%	封面、摘要、前言和目录
%---------------------------------------------------------------------

% 生成封面页
\maketitle

% 模板默认使用 \flushbottom,即底部平齐
% 效果更好,但可能出现 underfull \vbox 信息
% 以下命令用于抑制这些信息
\raggedbottom

\begin{abstract}
随着社交媒体在投资者行为中的影响日益增强,网络暴力作为一种极端负面情绪表达方式,其对金融市场系统性风险的潜在影响值得深入探讨。本文基于 Agent-Based 建模方法,构建融合社交传播机制与行为金融模型的人工股票市场系统,模拟网络暴力在社交网络中的扩散如何影响投资者行为决策,并进一步扰动市场价格与财富分布结构。
模型中引入攻击者、受害者与旁观者角色,通过攻击传播、情绪感染与行为反馈构建“情绪—行为—市场”闭环机制。市场结构采用连续双边报价(CDA)机制,投资者基于多维信号进行交易决策,网络暴力对其情绪偏差与参与概率产生动态干预。
通过多组仿真实验,本文系统分析了网络暴力机制对个体行为路径、群体财富演化以及市场效率的扰动效应。结果显示:网络暴力显著降低了散户交易活跃度与终期财富水平,造成群体内财富差距扩大;同时加剧了价格波动、提高了交易摩擦成本,并削弱了市场价格发现能力。
本文研究为理解社交媒体情绪传播如何通过行为反馈机制传导至市场系统提供了建模框架,也为未来建立金融情绪监管预警机制提供理论依据。

\end{abstract}

\begin{abstract*}
As social media increasingly shapes investor sentiment and decision-making, cyberbullying—an extreme form of negative emotional expression—has emerged as a potential source of systemic risk in financial markets. This study develops an Agent-Based artificial stock market model that integrates a social contagion mechanism to examine how the diffusion of cyberbullying through investor networks impacts individual trading behavior and, ultimately, market dynamics.
The model incorporates roles such as attackers, victims, and bystanders, and simulates a feedback loop among emotional contagion, behavioral responses, and market outcomes. A continuous double auction (CDA) mechanism governs market transactions, where heterogeneous agents make trading decisions based on fundamental, technical, and noise signals. The propagation of cyberbullying alters their emotional bias and participation probability over time.
Through a series of Monte Carlo simulations, the study quantitatively assesses the impact of cyberbullying on trading frequency, wealth distribution, price volatility, market liquidity, and informational efficiency. Results show that cyberbullying significantly reduces retail investor participation and terminal wealth, widens intra-group wealth inequality, increases market volatility, raises transaction costs, and weakens the market's ability to reflect fundamental values.
This research provides a novel framework for modeling the behavioral transmission of extreme social emotions into market-level disruptions and offers theoretical insights for future design of financial emotion monitoring and regulatory mechanisms.

\end{abstract*}

% 生成目录
\tableofcontents
% 生成图片清单
\listoffigures
% 生成表格清单
\listoftables

%---------------------------------------------------------------------
%	正文部分
%---------------------------------------------------------------------
\mainmatter

% 符号表
% 语法与 description 环境一致
% 两个可选参数依次为说明区域宽度、符号区域宽度
% 带星号的符号表(notation*)不会插入目录
% \begin{notation}[10cm]
%   \item[DFT] 密度泛函理论 (Density functional theory)
%   \item[DMRG] 密度矩阵重正化群 (Density-Matrix Reformation-Group)
% \end{notation}

% 建议将论文内容拆分为多个文件
% 即新建一个 chapters 文件夹
% 把每一章的内容单独放入一个 .tex 文件
% 然后在这里用 \include 导入,例如
%   \include{chapters/introduction}
%   \include{chapters/environments}


\chapter{导论}

\section{研究背景与意义}

随着社交媒体平台在公众生活中的日益普及,其在金融信息传播与投资者行为形成中的作用愈发重要。微博、股吧、雪球等平台已不再只是信息获取渠道,更成为投资者观点碰撞、情绪共鸣乃至群体决策的互动空间。社交媒体正在重塑市场主体的认知方式和行为路径,成为连接“信息流—情绪流—资金流”的关键载体。

近年来,学术界逐渐关注社交内容对金融市场的影响,尤其是社交情绪传播所引发的投资行为偏差与市场异动问题。已有研究表明,积极或消极情绪的集中释放往往会影响投资者的风险偏好与交易意愿,进而加剧价格波动、放大市场反馈效应,诱发“羊群行为”或“非理性繁荣”。

尽管相关文献在情绪构造与预测方面取得进展,但对于情绪在社交网络中如何扩散,并通过群体交互反馈影响市场的路径,仍缺乏系统刻画。特别是其中更极端的情绪表达形式——网络暴力(Cyberbullying)——作为攻击性、传染性并存的社会行为,尚未被有效纳入金融建模视野。

网络暴力的扩散具有非线性特征,常引发个体沉默、极端化或退场行为。这类行为不仅削弱市场多样性,也可能扰乱交易结构、降低定价效率,构成潜在的系统性风险。

另一方面,人工股票市场(Artificial Stock Market, ASM)作为行为金融与计算实验的重要工具,已广泛用于模拟投资者行为、制度变动与市场演化过程。ASM 可刻画异质性交易者在复杂环境下的交互行为,重现真实市场中的极端事件与财富集中等现象。然而,主流 ASM 模型往往未纳入社交机制,更缺乏对情绪传播与行为反馈链条的建模。

基于此,本文将网络暴力视为一种可传播、可反馈的社交情绪机制,嵌入 Agent-Based 人工市场系统中,系统探索其对投资者行为路径与市场效率的影响过程。该研究旨在弥补行为金融模型在“社交行为机制”刻画方面的空白,为理解社交平台风险如何传导至金融系统提供理论支撑与建模框架。

\section{研究内容与方法}

本文围绕网络暴力在社交网络中的传播机制及其对金融市场的影响路径展开,采用 Agent-Based 建模方法,构建集成“情绪传播—行为决策—市场反馈”机制的人工股票市场仿真系统。主要内容包括:

\begin{itemize}
  \item 构建具有攻击转化、情绪感染、沉默反应与治理反馈机制的网络暴力传播模型;
  \item 将传播机制嵌入人工市场系统,模拟网络暴力如何影响个体交易行为与市场状态;
  \item 设置对比实验,评估网络暴力机制对市场价格波动、流动性、财富分布等宏观结果的影响;
  \item 进一步分析不同机制参数(如监管强度、心理韧性、社交结构)对系统演化的调节作用。
\end{itemize}

研究方法综合使用了多主体行为建模、社交网络传播建模、连续双边报价市场设计、蒙特卡洛仿真、配对 T 检验与可视化分析,力求在行为层与市场层构建完整耦合机制。

\section{研究创新与不足}

本文在研究问题、模型结构与机制设计方面具有如下创新:

\begin{itemize}
  \item \textbf{问题视角新颖}:首次将网络暴力作为可传播、可反馈的社会行为机制引入人工市场模拟,关注其对个体行为与市场效率的双重影响;
  \item \textbf{机制设计复杂}:构建包含正反馈(攻击转化)与负反馈(举报、韧性恢复)的传播机制,刻画网络暴力在社交网络中的动态演化过程;
  \item \textbf{结构耦合完整}:集成“情绪传播—行为决策—市场反馈”链条,形成“情绪—行为—市场”的反馈闭环建模框架;
  \item \textbf{评估维度多元}:从个体层(情绪偏差、交易行为)、群体层(财富分布、行为极化)和系统层(流动性、信息效率)多维度开展分析。
\end{itemize}

同时,本文也存在若干局限性:

\begin{itemize}
  \item 模型参数主要基于理论设定,缺乏真实社交或市场数据校准;
  \item 网络结构为静态预设,未考虑社交连接随行为动态调整;
  \item 网络暴力情绪建模较为抽象,未纳入文本情绪、语义攻击强度等精细表达。
\end{itemize}

上述不足为后续模型拓展与实证研究提供了方向。

\section{章节安排}

本论文共分为五章,结构如下:

\begin{itemize}
  \item 第一章为导论,介绍研究背景、研究内容、创新点与论文结构;
  \item 第二章为文献综述,系统梳理人工市场建模、社交网络传播与情绪金融相关研究;
  \item 第三章为模型设定,介绍投资者行为模型、市场结构与网络暴力传播机制;
  \item 第四章为实验设计与仿真分析,对比有无网络暴力机制下系统演化特征;
  \item 第五章为结论与展望,总结主要发现,反思模型局限,并提出后续研究方向。
\end{itemize}

\chapter{文献综述}

随着社交媒体平台和计算实验方法的发展,投资者行为的建模路径与金融市场的模拟研究逐步从“理性—均衡—封闭”模型,转向更贴近现实的“异质—互动—开放”框架。人工股票市场(Artificial Stock Market, ASM)与Agent-Based建模(ABM)方法日益融合,逐步成为行为金融与制度实验研究的重要工具。同时,社交媒体成为投资者观点传播、情绪共振与信息扩散的重要载体,社交网络中的极端行为(如网络暴力)逐渐引发金融学界关注。

本章将从两个方面系统梳理已有研究:一是人工股票市场与基于Agent的行为建模路径,二是社交媒体中的情绪传播与投资者行为机制。最后总结已有文献的研究空白并明确本文的定位。

\section{人工股票市场与基于Agent的建模研究}

人工股票市场建模强调投资者之间的行为差异、市场结构的显性建模与制度环境的机制模拟,是金融复杂系统建模的重要方向。\textcite{chiarella2009impact}构建了包含基本面投资者、技术交易者与噪声交易者三类Agent的市场模型,并通过订单簿交易机制模拟价格形成过程,成功复现了波动聚集、厚尾收益等市场特征,是现代ABM金融市场模型的奠基之作。

随后大量研究在此基础上发展出更具行为异质性与制度复杂性的模型框架。\textcite{zhang2021financialengineering}指出,ABM的可解释性、灵活性与机制透明性使其成为大数据驱动的金融实验平台。\textcite{li2012tick}基于连续双边报价系统,研究了最小报价单位变化对流动性的影响;\textcite{zhou2023shortselling}与\textcite{xiong2020shortsellingsize}则从融券卖空机制出发,模拟提价规则与规模约束对市场稳定性的作用,验证了市场制度在微观结构层面的调节功能。

行为建模方面,\textcite{hu2022heterogeneous}将前景理论与认知偏差引入投资者策略形成过程,发现风险偏好异质性与主观概率扭曲是市场剧烈波动的重要内生源。\textcite{chen2020herding}构建了羊群行为的协同扩散模型,说明个体模仿机制在推动集体非理性与市场异常中的作用;\textcite{zou2021chaotic}进一步指出,投资者反馈行为与策略演化可诱发市场的混沌波动。

近期研究趋势逐渐转向多层结构耦合与宏观系统稳定性评估。\textcite{zhangyi2022micromodel}设计了基于元模型的异质交易行为仿真框架,集成了交易者策略、市场机制与监管政策;\textcite{liangrui2022starboard}以科创板为案例,模拟交易机制改革对波动性与稳定性的双重影响。\textcite{wei2023t0}与\textcite{wei2021liquiditycrisis}分别研究了T+0交易制度与流动性踩踏机制,强调市场制度变化对系统风险演化路径的引导作用。

在国际研究方面,\textcite{gatti2020learningcovid}将COVID-19视作前所未有的外部冲击,通过ABM框架模拟投资者如何调整信念与行为,从而影响市场的结构性转变。\textcite{dawid2023behavioralheterogeneity}进一步将行为异质性视为金融市场宏观动态演化的核心变量,提出多群体间策略学习、社会互动与制度反馈的整合建模路径。

总体来看,ASM研究已具备较强的行为表达能力与制度实验基础,但在建模结构中尚缺乏投资者间社交传播、情绪传染与极端行为反馈的机制集成。

\section{社交媒体、投资者情绪与金融市场}

社交媒体在投资者情绪形成、信息扩散与行为反馈中扮演着愈发重要的角色。平台上的情绪表达不但反映市场预期,还可能通过群体模仿与社交传播引发集体行为偏差与市场异动。\textcite{cheng2020mediaemotion}基于微博数据构建了涨跌情绪指数,发现负面情绪与市场收益存在显著Granger因果关系,具有领先性预测能力。\textcite{shi2021attentionemotion}整合雪球、股吧与新闻评论构建情绪强度与关注度指标,指出社交关注是交易量与换手率波动的核心驱动因子。

\textcite{li2019emotionreturn}利用股吧评论文本提取投资者情绪指数,并结合Fama-MacBeth回归验证其对收益率的显著解释力,是社交数据情绪测度研究的重要代表。\textcite{liu2020abmreview}从理论角度指出,未来的金融行为建模应整合个体情绪状态、社交传播结构与市场微观机制,构建多层嵌套的耦合系统。

在建模路径上,\textcite{vanfossan2020abmsocial}首次将社交媒体情绪传播机制嵌入ABM市场结构中,构建了“市场—媒体—社交网络—投资者”多层交互路径,通过伪Agent设计模拟信息与情绪在不同媒介中的传播与反哺,开创了情绪传播与行为决策的耦合建模路径。

然而,已有文献多聚焦于情绪指数构建与传播路径模拟,较少涉及极端社交行为(如攻击性言论、群体压制)在市场系统中的反馈路径建模,缺乏对网络暴力这类情绪突发事件的行为干预与制度治理模拟。

\section{文献评述与本研究定位}

综上,已有文献在人工市场结构设计、异质性行为建模与社交媒体情绪量化等方面积累了丰富成果:

\begin{itemize}
  \item 在建模方法上,ASM结构已发展出多策略融合、制度调节与风险模拟等功能性机制;
  \item 在社交研究中,投资者情绪与市场行为之间的统计关系已得到充分验证;
  \item 部分研究开始探索社交媒体与金融市场之间的行为耦合机制,为社交传播模型与ABM融合奠定基础。
\end{itemize}

但在研究深度与机制集成方面仍存在明显空白:

\begin{itemize}
  \item 当前模型普遍缺乏对\textbf{网络暴力等极端负面情绪机制}的建模,无法解释攻击性情绪如何扰动投资者行为并反馈至市场系统;
  \item 多数研究未能构建\textbf{“社交传播—行为偏差—市场反馈”}的完整闭环,难以支持机制识别与政策实验;
  \item \textbf{情绪治理变量}(如举报、韧性成长、监管者介入)缺位,阻碍了模型在公共政策与行为干预方面的应用拓展。
\end{itemize}

本文拟构建一个集成网络暴力传播机制的多主体人工市场系统,在异质交易行为与连续双边市场结构基础上,引入社交网络结构、攻击传播模型与反馈调节机制,系统模拟“情绪—行为—市场”三层动态反馈路径,并评估情绪治理机制的调节效应,为金融系统中极端社交情绪的建模与治理提供理论工具。

\chapter{模型设定}




\section{模型总体框架}

本研究构建了一个集成社交传播机制与行为金融模型的人工股票市场系统,旨在模拟网络暴力这一极端情绪事件如何通过社交网络影响投资者行为,并进一步扰动市场运行机制。整体模型以Agent-Based建模为基础,包含投资者行为模块、市场交易结构模块、网络暴力传播模块和实验分析模块四大核心组成部分,四者在系统中相互耦合,构成“情绪—行为—市场”闭环反馈结构。

在该系统中,投资者被建模为具有异质偏好、有限理性与社交关系的自主Agent,分为机构投资者与散户两类。两类Agent均可基于基本面预期、趋势信号与随机扰动做出交易决策,但在行为风格与受情绪影响程度上存在显著差异,特别是散户更易受到网络暴力影响而表现出沉默或激进等行为偏差。

市场交易结构采用连续双边报价(Continuous Double Auction, CDA)机制,通过订单簿系统撮合市价单与限价单形成成交价格。基础资产价格由布朗运动驱动,模拟市场的外部波动环境。所有Agent的交易意愿以订单形式提交至市场,经过撮合成交,最终影响市场价格演化与财富流动路径。

网络暴力传播模块在散户投资者之间建立社交网络,模拟在社交媒体语境下攻击性言论的传播过程。该模块引入攻击者、受害者、旁观者三类角色转化机制,并设计了情绪感染(正反馈)与系统治理(负反馈)机制。具体传播过程遵循以下逻辑:攻击者选择邻居中的异见者进行言语攻击,受害者若暴露程度超过阈值则转为沉默或反击者,同时系统可通过举报、监管干预或个体心理韧性成长抑制传播范围。

实验分析模块用于对模型结果进行系统分析,主要从两个角度进行分析:一是投资者群体的财富状态变化,二是市场的总体状态(如流动性、波动性等)。该模块通过对两类投资者(散户与机构)财富变化的追踪,分析网络暴力情绪如何影响投资者的财富分布;同时,分析市场的流动性、波动率等指标,探讨网络暴力是否对市场质量产生显著影响。实验分析模块通过对比不同情境下的结果,验证网络暴力对市场的实际影响。

模型整体运行流程如下:在每个仿真时间步,市场随机激活部分投资者进行决策;情绪传播模块并行更新网络暴力状态;投资者依据当前情绪状态、策略参数与市场信息做出交易决策;订单经由市场结构撮合成交;价格、资产与情绪状态更新,进入下一个时间步。通过多轮模拟与对比实验,可以观察网络暴力机制在投资者行为、市场稳定性与财富演化中的影响路径。


\begin{figure}[h]
    \centering
    \includegraphics[width=0.8\textwidth]{image/3-1_structure.pdf}
    \caption{模型整体架构}
    \label{fig:architecture}
\end{figure}

图\ref{fig:architecture} 展示了模型的总体结构框架,四个核心子系统通过状态变量与行为函数实现交互,形成多层次、跨模块的系统性耦合。








\section{市场结构模块}

市场结构模块模拟了股票市场中的交易机制,特别是订单簿管理、订单匹配以及价格生成过程。本部分将首先讨论订单和订单簿的功能与实现,随后分析市场如何通过订单簿撮合机制生成价格并更新市场状态。

\subsection{订单与订单簿}


市场交易通过订单簿(Order Book)进行管理,订单簿包含市场上所有的买单和卖单,按照价格优先和时间优先的规则进行排列,并通过撮合机制完成交易。在本模型中,投资者可以提交两种类型的订单:市价单(Market Order)和限价单(Limit Order)。

市价单是指投资者希望以市场上当前最优价格立即成交的订单,买单市价单与当前最优卖单匹配,卖单市价单与当前最优买单匹配。限价单则是投资者愿意在特定价格或更好价格下买入或卖出资产的订单。限价单会根据价格优先和时间优先的规则排队,价格优先确保更高的买单和更低的卖单优先成交,时间优先则确保在价格相同的情况下,先提交的订单优先成交。

限价单的提交和撮合过程如下:

1. 买单限价单:在订单簿中,买单限价单按价格从高到低排列。只有当卖单的价格满足买单的价格时,买单才能成交。若价格相同,按时间优先规则进行撮合。

2. 卖单限价单:卖单限价单按价格从低到高排列,买单的价格需要等于或高于卖单的价格才会发生成交。若价格相同,按照提交时间优先。

限价单的核心是它为市场提供了深度,当市场价格变动时,挂单价格会进行相应的调整。只有当市场价格到达某个限价单的价格时,订单才会被执行。

另一方面,市价单与限价单的关系较为复杂,市价单会根据当前订单簿中的最优卖单(对于买单市价单)或最优买单(对于卖单市价单)立即执行。市价单的优先性使其能够快速成交,而限价单则在价格条件不满足时等待成交。若市价单的数量大于市场中最优对手单的数量,未成交的部分会转为限价单,继续在订单簿中等待匹配。

市场中的买单和卖单队列由订单簿进行管理。订单簿通过使用堆(heapq)数据结构实现对订单的优先级排序。买单按价格从高到低(负数价格)排序,卖单按价格从低到高排序,确保买卖双方能够根据最优价格进行交易。


表 \ref{tab:buy_orderbook} 展示了一个典型的买单订单簿示意表,其中包括了买单的订单编号、交易者ID、买入数量、价格、时间戳和订单状态等信息。



\begin{table}[htbp]
    \centering
    
    
    \begin{tabularx}{\textwidth}{@{} *{6}{>{\centering\arraybackslash}X} @{}}
    \toprule
    订单编号 & 交易者ID & 买入数量 & 价格 & 时间戳 & 订单状态 \\ \midrule
    1 & 101 & 100 & 99.95 & 1 & 待处理   \\
    2 & 102 & 200 & 99.90 & 2 & 待处理   \\
    3 & 103 & 150 & 99.85 & 3 & 已成交  \\
    4 & 104 & 100 & 99.80 & 4 & 待处理  \\
    5 & 105 & 50  & 99.75 & 5 & 已取消 \\ \bottomrule
    \end{tabularx}
    \caption{买单订单簿示意表}
    \label{tab:buy_orderbook}
\end{table}

在实际市场中,订单簿会随时间的推移动态变化,特别是当市场价格波动时,投资者提交的限价单会被重新排序,市价单则根据当前市场价格立即成交。每个时间步,模型会根据最新的市场状态和订单簿情况进行更新,形成价格和市场行为的反馈循环。





\subsection{价格生成机制与市场状态更新}

在本模型中,基础价格由几何布朗运动(Geometric Brownian Motion, GBM)生成,而市场价格则通过市场中的订单簿和交易活动动态生成。基础价格反映了市场资产的长期趋势和短期波动,而市场价格则是由市场参与者的交易行为和订单簿状况决定的。基础价格和市场价格的相互作用是市场价格动态更新的核心。

1. 基础价格生成

基础价格的生成依赖于几何布朗运动模型,这是金融市场中常用的价格生成模型。几何布朗运动模型的价格更新公式如下:

\begin{equation}
    p_{t+1} = p_t \exp\left( \mu \Delta t + \sigma \epsilon_t \sqrt{\Delta t} \right)
\end{equation}

其中,\( p_t \) 是当前价格,\( \mu \) 是漂移项,表示资产的长期趋势,\( \sigma \) 是波动率,表示价格的波动幅度,\( \epsilon_t \) 是标准正态分布的随机扰动项,\(\Delta t\) 是时间步长。

几何布朗运动能够模拟市场中常见的随机波动,反映资产价格的随机性和市场的不确定性。基础价格是市场中的理论价格,它不直接反映交易者的决策和市场供需情况,而是一个由宏观因素和市场波动性驱动的参考价格, 在本模型中是交易者进行基本面分析时的重要参照。

2. 市场价格生成

市场价格是由市场中的买单和卖单通过订单簿的撮合机制决定的。每个市场时间步,投资者通过市价单或限价单提交订单,市场则根据最优买单和最优卖单的价格进行撮合。当买单价格大于或等于卖单的最优价格时,订单会成交,成交价格即为市场价格。
如果在某一时间步没有成交,则市场价格将根据当前最优买单和最优卖单的价格更新,具体公式如下:

\begin{equation}
    \text{Market Price} = 
    \begin{cases} 
    \text{last\_price} & \text{如果上一时间步有交易发生} \\
    \frac{\text{best\_bid} + \text{best\_ask}}{2} & \text{如果上一时间步没有交易发生}
    \end{cases}
\end{equation}
在市场价格的生成过程中,市价单和限价单之间的相互作用起到了关键作用。市价单是根据当前市场最优对手单立即成交的订单,而限价单则需要在特定的价格范围内等待成交。市价单的成交价格通常会成为市场的最新价格,反映了市场供需的实时状况。





3. 市场状态更新

市场价格更新的基本过程如下:

- 市价单成交:市价买单会与当前最优卖单匹配,而市价卖单会与当前最优买单匹配。当市价单和限价单匹配时,成交价格即为最优买单或卖单的价格。
   
- 订单簿更新:每次成交后,订单簿中的相应订单会被移除,剩余的订单根据价格和时间优先规则重新排列。如果价格发生变化,订单簿将更新,确保挂单的顺序正确。

- 市场价格更新:市场成交价格会被用作当前市场的参考价格,并更新市场价格。通过对成交价格的跟踪,市场价格会反映出当前交易者的行为和市场的供需状况。

市场状态的更新不仅仅是价格的变化,还包括市场的流动性、深度和交易量等因素。在每个时间步,市场的流动性和深度都会随着订单簿的变化而更新。流动性反映了市场能够在不显著影响价格的情况下吸纳大规模交易的能力,而市场深度则表示在特定价格区间内存在的买单和卖单的数量。

市场的买卖价差(Bid-Ask Spread)可以通过以下公式计算:

\begin{equation}
    \text{Bid-Ask Spread} = \text{best\_ask} - \text{best\_bid}
\end{equation}

市场的深度表示市场中各个价格级别的买单和卖单的总数量。市场深度可以通过以下公式进行计算:

\begin{equation}
    \text{Market Depth} = \sum_{i=1}^{n} (\text{buy\_quantity}_i + \text{sell\_quantity}_i)
\end{equation}

其中,\( \text{buy\_quantity}_i \) 和 \( \text{sell\_quantity}_i \) 分别表示在档位 \( i \) 上的买单和卖单数量,\( n \)代表盘口档数(一般为5档)。市场深度可以反映市场的流动性,即在不显著影响价格的情况下能够吸纳的订单量。



\chapter{模拟过程}
这里是模拟过程

\chapter{结论与展望}

本文基于 Agent-Based 建模方法,构建了一个融合网络暴力传播机制的人工股票市场模拟系统,系统性研究了社交媒体极端言论如何通过影响投资者行为进一步扰动市场运行机制。本文将投资者异质性行为建模、连续双边市场结构和情绪传播机制集成在统一框架内,刻画了“情绪—行为—市场”之间的闭环耦合关系,并通过多组仿真实验探讨了网络暴力机制的经济后果。

\section{主要研究结论}

结合模型设定与实证结果,本文得出以下核心结论:

\begin{itemize}
  \item \textbf{网络暴力显著扰动投资者行为路径}。在情绪传播机制影响下,散户群体中部分个体因受到攻击而表现出交易沉默、撤单意愿增强等行为偏差,导致其长期交易参与度与收益能力显著下降。
  
  \item \textbf{攻击状态与财富演化路径存在显著差异}。被攻击个体的终期财富显著低于未被攻击者,而未被攻击者甚至可能因“幸存效应”受益,反映出情绪干扰机制下的群体内部分化效应。
  
  \item \textbf{网络暴力加剧了财富不平等与市场割裂}。散户群体的内部基尼系数在网络暴力情景下显著上升,说明市场呈现出“强者愈强、弱者沉默”的结构性分化趋势。
  
  \item \textbf{网络暴力机制导致市场效率系统性下降}。实证结果表明,情绪扰动不仅加剧了市场波动(波动率提升),也提高了交易摩擦成本(买卖价差扩大),同时弱化了价格发现能力(价格偏离度上升),表现出多维度的市场效率损害。
\end{itemize}

\section{研究局限与改进方向}

尽管本文提出了一个较为完整的情绪行为市场耦合建模框架,但仍存在以下局限:

\begin{itemize}
  \item \textbf{投资者行为建模仍比较简化}:情绪偏差与决策过程虽然被区分为机构与散户,但尚未引入真实认知更新、演化学习或舆情反馈等更复杂机制。
  
  \item \textbf{网络结构固定,未考虑社交演化}:目前网络暴力传播模块依赖于预设的小世界网络,尚未模拟攻击过程中社交边的变化、群体极化或同温层效应等更复杂的动态社交演化过程。
  
  \item \textbf{模型验证缺乏真实数据对照}:由于当前主要采用模拟实验方式验证机制效果,尚未结合微博、雪球等平台上的真实情绪数据进行定量对照分析。
\end{itemize}

未来研究可从以下方向对模型进一步完善:

\begin{itemize}
  \item 引入舆情文本挖掘模型,将社交平台真实攻击性言论嵌入情绪传播模块,增强模型的实证效度;
  \item 构建动态社交网络,使攻击行为与社交连接产生联动,从而模拟“回声室效应”“社群极化”等真实传播现象;
  \item 增加监管与平台治理机制,如举报、封禁等手段,评估治理机制对市场效率与行为稳定性的恢复作用。
\end{itemize}

\section{现实启示与未来展望}

本文的研究不仅为人工市场模拟引入了社交语境中的极端情绪机制,也为理解社交平台与金融市场之间的互动路径提供了一种结构性解释视角。主要现实启示如下:

\begin{itemize}
  \item \textbf{情绪管理是金融市场稳定的重要一环}。网络暴力等非理性言论的传播不仅扰乱信息环境,也可能诱发投资者集体行为偏差,从而带来价格异常与市场系统性风险。
  
  \item \textbf{交易平台与社交平台需联动治理}。证券监管部门与社交平台方可考虑建立联合风控机制,将舆情波动作为异常交易风险的前置信号,构建情绪-行为联合预警系统。
  
  \item \textbf{市场系统建模需纳入情绪反馈与社交传播机制}。未来基于 ABM 的金融市场研究应当更充分地吸收行为金融与社交网络理论成果,模拟真实社交行为如何嵌入市场结构之中。
\end{itemize}

综上所述,本文构建了一个情绪传播与金融行为交互的实验框架,为未来金融-社交交叉机制研究提供了基础样本与可推广的分析逻辑。希望本研究能够为理解极端情绪事件的系统性后果与市场治理路径提供新的模型工具与分析视角。


%---------------------------------------------------------------------
%	参考文献
%---------------------------------------------------------------------

% 生成参考文献页
\printbibliography

%---------------------------------------------------------------------
%	致谢
%---------------------------------------------------------------------

\begin{acknowledgement}
  
  
时光荏苒,转眼已至本科学业的尾声。四年前的盛夏,我有幸进入南京大学工程管理学院计算机金融工程实验班,开始了这段融合技术与金融、理论与实践的学习旅程。在这里,我不仅拓展了视野、锤炼了能力,也逐渐明确了自己的未来方向

首先,我要衷心感谢我的论文指导老师李心丹老师和孙煦初老师。在选题讨论、模型设计、实验调参乃至论文撰写的每一个阶段,老师都给予了我悉心的指导与启发。在我思路迷茫、表达混乱时,老师总能以严谨的逻辑与宽容的态度,引导我重新理清问题。感谢老师不仅教会我如何“写一篇论文”,更让我体会到科研训练背后严谨求实的精神。

感谢工程管理学院和其他院系给我上过课的各位老师四年来在课程教学、项目指导与学业支持方面给予的帮助。黄卫华老师的幽默口音,许慨老师“地狱难度”的口语表达作业,陈莹老师课上的企业模拟实训,肖斌卿老师的“放电影”环节,都是琉璃般的时间碎片值得珍存。

感谢我的家人,是你们始终无条件地支持我所有的决定,无论是专业选择、职业规划还是论文方向。你们的理解、信任与陪伴,是我不断前行的底气与动力。

也要感谢身边的朋友与同学,在我论文写作过程中提供了各种帮助与陪伴。每一次夜深时的调试、每一次讨论后的顿悟,都是这段旅程中珍贵的记忆。

最后,愿我不负这段本科生涯的成长与磨炼,不负南大“诚朴雄伟,励学敦行”的精神,以理性与温度面对未来的挑战,在真实世界中继续建构属于自己的解法。
  
\end{acknowledgement}

%---------------------------------------------------------------------
%	学术简历
%---------------------------------------------------------------------

% 详见手册中"成果列表"一节
% \njuchapter{学术成果}
% \njupaperlist[攻读博士学位期间发表的学术论文]{preskill2018}

%---------------------------------------------------------------------
%	附录部分
%---------------------------------------------------------------------

% 附录部分使用单独的字母序号
\appendix

% 可以在这里插入补充材料

\chapter{模型部分参数}
\label{appendix:params}

\begin{table}[h]
    \renewcommand{\arraystretch}{1.4}
    \centering
    \begin{tabular}{@{}lclp{8.5cm}@{}}
    \toprule
    \textbf{模块} & \textbf{符号} & \textbf{默认值} & \textbf{参数说明} \\
    \midrule
    \multicolumn{4}{@{}l}{\textbf{市场参数}} \\
    & \( T \) & 50000 & 最大仿真步数 \\
    & \( \alpha \) & 0.1 & 每步激活的投资者比例 \\
    & \( p_0^f \) & 300.0 & 初始基础价格 \\
    & \( \sigma_f \) & 0.001 & 基础价格波动率 \\
    & \( \mu_f \) & 0.00002 & 基础价格漂移项 \\
    \midrule
    \multicolumn{4}{@{}l}{\textbf{投资者参数}} \\
    & \( N \) & 1000 & 投资者总数 \\
    & \( \tau_f \) & 100 & 投资期参考长度 \\
    & \( \sigma_n \) & 0.01 & 预期收益的噪声项标准差 \\
    
    & \( \sigma_s \) & 0.5 & 策略权重的波动程度 \\
    & \( \eta_0 \) & 0.05 & 情绪偏差初始化标准差 \\
    & \( \mu_e \) & 1.0 & 情绪权重对数正态分布均值 \\
    \midrule
    \multicolumn{4}{@{}l}{\textbf{网络暴力参数}} \\
    & \( d \) & 6 & 社交网络平均连接度 \\
    
   
    & \( \theta_x \) & 0.005 & 曝光阈值(认定被攻击) \\
    
    & \( \kappa \) & 0.999 & 曝光自然恢复因子 \\
    & \( \kappa_e \) & 0.8 & 情绪极端性衰减因子 \\
    & \( \pi_b \) & 0.05 & 网暴感染概率 \\
   
    & \( \pi_s \) & 0.5 & 被攻击沉默概率基线 \\
    
    \midrule
    \multicolumn{4}{@{}l}{\textbf{市场反馈机制参数}} \\
    & \( T_r \) & 100 & 监管者巡视周期 \\
    & \( T_{rc} \) & 100 & 惩罚冷却时间 \\
    & \( \beta_r \) & 0.1 & 举报成功概率 \\
    & \( T_{rrc} \) & 100 & 举报惩罚冷却时间 \\
    & \( s_{\max} \) & 0.95 & 最大沉默概率 \\
    \bottomrule
    \end{tabular}
    \caption{模型核心参数与默认值}
    \label{appendix:params}
    \end{table}
    


% 完工
\end{document}
