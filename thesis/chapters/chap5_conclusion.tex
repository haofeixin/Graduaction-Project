
\chapter{结论与展望}

本文基于 Agent-Based 建模方法,构建了一个融合网络暴力传播机制的人工股票市场模拟系统,系统性研究了社交媒体极端言论如何通过影响投资者行为进一步扰动市场运行机制。本文将投资者异质性行为建模、连续双边市场结构和情绪传播机制集成在统一框架内,刻画了“情绪—行为—市场”之间的闭环耦合关系,并通过多组仿真实验探讨了网络暴力机制的经济后果。

\section{主要研究结论}

结合模型设定与实证结果,本文得出以下核心结论:

\begin{itemize}
  \item \textbf{网络暴力显著扰动投资者行为路径}。在情绪传播机制影响下,散户群体中部分个体因受到攻击而表现出交易沉默、撤单意愿增强等行为偏差,导致其长期交易参与度与收益能力显著下降。
  
  \item \textbf{攻击状态与财富演化路径存在显著差异}。被攻击个体的终期财富显著低于未被攻击者,而未被攻击者甚至可能因“幸存效应”受益,反映出情绪干扰机制下的群体内部分化效应。
  
  \item \textbf{网络暴力加剧了财富不平等与市场割裂}。散户群体的内部基尼系数在网络暴力情景下显著上升,说明市场呈现出“强者愈强、弱者沉默”的结构性分化趋势。
  
  \item \textbf{网络暴力机制导致市场效率系统性下降}。实证结果表明,情绪扰动不仅加剧了市场波动(波动率提升),也提高了交易摩擦成本(买卖价差扩大),同时弱化了价格发现能力(价格偏离度上升),表现出多维度的市场效率损害。
\end{itemize}

\section{研究局限与改进方向}

尽管本文提出了一个较为完整的情绪行为市场耦合建模框架,但仍存在以下局限:

\begin{itemize}
  \item \textbf{投资者行为建模仍比较简化}:情绪偏差与决策过程虽然被区分为机构与散户,但尚未引入真实认知更新、演化学习或舆情反馈等更复杂机制。
  
  \item \textbf{网络结构固定,未考虑社交演化}:目前网络暴力传播模块依赖于预设的小世界网络,尚未模拟攻击过程中社交边的变化、群体极化或同温层效应等更复杂的动态社交演化过程。
  
  \item \textbf{模型验证缺乏真实数据对照}:由于当前主要采用模拟实验方式验证机制效果,尚未结合微博、雪球等平台上的真实情绪数据进行定量对照分析。
\end{itemize}

未来研究可从以下方向对模型进一步完善:

\begin{itemize}
  \item 引入舆情文本挖掘模型,将社交平台真实攻击性言论嵌入情绪传播模块,增强模型的实证效度;
  \item 构建动态社交网络,使攻击行为与社交连接产生联动,从而模拟“回声室效应”“社群极化”等真实传播现象;
  \item 增加监管与平台治理机制,如举报、封禁等手段,评估治理机制对市场效率与行为稳定性的恢复作用。
\end{itemize}

\section{现实启示与未来展望}

本文的研究不仅为人工市场模拟引入了社交语境中的极端情绪机制,也为理解社交平台与金融市场之间的互动路径提供了一种结构性解释视角。主要现实启示如下:

\begin{itemize}
  \item \textbf{情绪管理是金融市场稳定的重要一环}。网络暴力等非理性言论的传播不仅扰乱信息环境,也可能诱发投资者集体行为偏差,从而带来价格异常与市场系统性风险。
  
  \item \textbf{交易平台与社交平台需联动治理}。证券监管部门与社交平台方可考虑建立联合风控机制,将舆情波动作为异常交易风险的前置信号,构建情绪-行为联合预警系统。
  
  \item \textbf{市场系统建模需纳入情绪反馈与社交传播机制}。未来基于 ABM 的金融市场研究应当更充分地吸收行为金融与社交网络理论成果,模拟真实社交行为如何嵌入市场结构之中。
\end{itemize}

综上所述,本文构建了一个情绪传播与金融行为交互的实验框架,为未来金融-社交交叉机制研究提供了基础样本与可推广的分析逻辑。希望本研究能够为理解极端情绪事件的系统性后果与市场治理路径提供新的模型工具与分析视角。
