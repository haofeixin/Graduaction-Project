\chapter{文献综述}

\section{社交媒体与金融市场相关研究}

近年来,随着社交媒体平台在投资者群体中的广泛使用,其对金融市场的影响成为学术界与业界关注的焦点。社交媒体不仅改变了投资者获取信息的方式,还重塑了舆论形成、情绪传染和投资决策的机制。相关研究主要集中于两个方向:一是探讨社交媒体生成的信息内容、投资者情绪如何预测市场走势;二是研究社交网络结构、情绪反馈机制如何在市场中形成羊群行为、泡沫或极端波动。

\textcite{bohorquez2024social} 构建了一个基于多层社交网络的金融市场ABM模型,试图模拟社交媒体环境下市场泡沫的形成机制。该研究在传统Lux-Marchesi框架的基础上,引入了社交结构中社区层级的划分与信息过滤路径,通过社交媒体模拟群体间信息选择性暴露(selective exposure)与“回音室效应”(echo chamber effect)的双重作用。模型结果显示,社交互动的增强会显著放大投资者意见的一致性,形成价格的非理性上升,从而诱发泡沫。该研究的重要贡献在于将社交传播建模与行为金融建模有机结合,但其网络结构为静态预设,尚未涵盖动态扩散或负面信息的传播路径。

在实证研究方面,\textcite{shi2020social} 利用中国A股市场中投资者论坛、雪球评论、财经新闻等多平台社交媒体数据,构造了投资者关注度、新闻情绪指数、社交互动强度等多维度变量,并将其引入VAR模型中,考察其与股市收益、波动之间的动态关系。研究发现,投资者情绪具有一定的市场预测能力,尤其是在情绪较为一致时更能引发市场短期价格反应。值得注意的是,不同平台的情绪信号存在时滞和偏差:股吧更易形成情绪过热,新闻评论则具有滞后反映。该研究提供了将多平台社交信号融合建模的实证基础,然而其模型未深入探讨情绪传播的方向性与结构影响。

\textcite{cheng2019emotion} 基于新浪微博大数据构建了涨跌情绪指数,将每日微博中与股市相关的文本内容进行分词与情感分析,提取乐观与悲观情绪占比作为指标,并将其与证券市场指数进行关联分析。研究发现,投资者情绪不仅与当日市场涨跌同步变化,在极端市场环境下还具有前瞻性特征。尤其在熊市环境下,情绪的负面反应更为迅速、强烈,形成情绪放大效应,说明社交媒体情绪可能通过非对称路径影响市场波动。但该研究未进一步探讨微观个体如何响应这些情绪信息。

\textcite{li2018emotion} 则进一步建立了理论与实证结合的情绪—收益通道。他们提出一个包含理性投资者与情绪交易者的市场结构模型,并使用东方财富股吧中海量发帖文本构建投资者情绪变量。在实证检验中,使用Fama-MacBeth双重回归法评估情绪对未来股票收益的预测能力,结果发现,情绪变量不仅显著预测未来收益,其影响力随“情绪投资者比例”的提高而增强。这一结果佐证了行为金融中的噪声交易理论,即非理性交易者通过群体性偏差扰动市场价格。该研究的优势在于理论建模与数据构造相匹配,但其情绪变量仍基于关键词匹配,缺乏动态情绪传播的建模框架。

综上所述,现有关于社交媒体对金融市场影响的研究主要聚焦在“情绪变量构造”与“预测效能分析”两个层面。相关文献普遍确认社交平台上的情绪指标对市场波动具有一定预测性,并表现出在市场极端情形下的非对称放大效应。然而,这类研究多数仍停留在统计相关层面,缺乏对情绪如何在人群中传播、如何通过微观行为反馈至市场的因果机制建模。此外,对“负面情绪”尤其是攻击性、排他性强的网络暴力行为的刻画也尚属空白。

本文在上述研究基础上进一步细化研究对象,将“网络暴力”作为社交媒体中的一种极端负面舆情现象进行机制性建模。不同于一般情绪信号,网络暴力具有攻击传播性、群体极化性与行为抑制性,其在社交网络中呈现出强烈的非线性扩散与正反馈特征。本文拟引入动态社交图结构与反馈机制,建构“攻击—沉默—激化”的传播路径,进一步探索其在投资行为与市场层面上的传导逻辑。这种机制建模为现有情绪金融与ABM研究提供了新路径,也为理解社交媒体风险向金融市场渗透提供了理论支持。


\section{人工股票市场相关研究}

人工股票市场(Artificial Stock Market, ASM)是一种模拟现实金融市场的计算实验系统,核心思想是基于Agent-Based Modeling(ABM)框架构建具有异质性、有限理性与适应性学习能力的投资者群体,并通过其相互作用演化出价格波动、交易行为与市场结构特征。相较于传统金融理论假设投资者完全理性并服从一致预期,ASM 模型强调市场是复杂系统,涌现行为由微观规则驱动。

\textcite{axtell2022agent} 从整体回顾了ABM在经济与金融中的发展历程,指出ABM特别适用于刻画非平稳系统中的非线性反馈、自组织行为以及制度实验。该研究回顾了Santa Fe人工市场模型的演进,认为该框架开启了用计算实验代替数理解析来研究市场动态的路径,具备解释市场集聚波动、长尾分布与非对称反应等特征的能力。作者还强调ABM模型的优势在于可实现微观异质性与宏观现象之间的桥接,为金融市场系统性风险研究提供了技术基础。

在中国学术界,\textcite{liu2020review} 对国内基于Agent的金融市场模型研究进行了系统综述,提出金融ABM模型可分为价格发现机制类、信息传播与学习机制类、市场结构与监管机制类等多个分支。文献指出,尽管ABM具备很强的行为表达能力,但其建模标准尚未统一,参数设定缺乏统一规范,模型验证方法仍以实验回测为主,存在一定主观性。作者建议未来应加强与真实市场数据的耦合,以及多源异构Agent的交互机制设计。

\textcite{gao2020wealth} 设计了一个引入演化学习与模仿机制的ASM模型,考察不同类型投资者在有限信息条件下如何调整策略,并最终影响市场财富分布。研究发现,市场中存在“赢家通吃”现象,早期收益较好的投资者更易成为模仿对象,导致其策略进一步被复制,形成路径依赖的财富集聚。同时,社会网络结构显著影响模仿路径:在集中网络中更易形成财富马太效应,在分散结构中则更利于多样化策略生存。该研究较为系统地模拟了微观行为演化如何通过模仿机制影响宏观结构,但尚未纳入情绪变量对行为调整路径的调控作用。

\textcite{chen2021herding} 将协同羊群行为引入人工股票市场中,模拟在不完全信息条件下,投资者如何在信念趋同时发生联动交易行为。模型结果显示,当投资者预期收敛速度过快或信号准确性过高时,市场更容易形成集体非理性行为,进而触发价格剧烈波动甚至市场崩盘。这一研究验证了“羊群行为”与市场系统性波动之间的因果关系,也突显出微观预期形成机制对宏观市场稳定性的重要影响。但该模型未构建明晰的信息传播路径,无法反映如社交媒体等现实平台上的非结构性意见形成过程。

此外,部分研究还尝试将行为金融理论中的偏好异质性与风险感知机制引入ABM框架中,形成情绪-认知联动模型。例如,一些学者通过设置“噪声交易者”与“理性交易者”并存结构,模拟恐慌行为如何通过市场回撤诱发群体抛售;另一些模型则引入投资者信心指数或情绪阈值作为行为切换机制,形成阶段性多稳定结构。这类模型在解释市场崩盘与反弹周期性上具有优势,但大多数尚未引入社交机制作为情绪扩散媒介。

综上所述,人工股票市场研究已建立起以ABM为核心的方法体系,能够从微观行为出发模拟市场的集体动态、价格形成机制以及财富演化路径,尤其在非线性波动、极端事件重现、制度测试等方面展现出独特优势。但当前研究仍存在两个空白:一是对外部情绪变量如何通过社交结构作用于个体行为的机制性建模仍较少;二是对负面情绪如恐慌、攻击性言论等如何在Agent系统中传播并反馈至市场层面的分析不足。

本文在前人研究的基础上引入社交网络结构与负面情绪传播机制,模拟网络暴力事件如何通过信息网络与情绪传导路径影响个体行为决策,并最终集体性反馈至市场价格形成与财富差异演化过程。这一拓展不仅丰富了人工市场模型的外部扰动机制设计,也为ABM模型中引入社会心理变量提供了实践模板。


\section{文献评述与研究空白分析}

从已有文献可以看出,当前金融学界与计算社会科学领域对社交媒体与金融市场之间关系的研究已取得初步进展,并在两个相对独立的方向上形成了较为系统的知识体系。一方面,社交媒体相关研究主要聚焦于情绪挖掘与预测能力验证,强调舆情数据的建模方法、市场反应特征及其短期预测能力;另一方面,人工股票市场(ASM)作为行为金融的建模平台,致力于还原现实市场中的非线性波动、行为异质性和系统性风险。

然而,尽管两者都承认行为与舆论对市场的影响,但彼此之间的整合仍存在显著不足。大多数社交媒体研究仍停留在数据驱动的经验分析层面,尽管可以通过文本挖掘技术提取情绪变量,并发现其与市场走势之间的相关性,但在机制建构上缺乏“因果链”式的建模逻辑。换言之,这些研究往往无法解释“情绪是如何传播的”“谁更容易被影响”“个体反应是否具有结构性异质”等关键问题,导致预测虽准,但解释力弱。

与此同时,ASM 方向的研究在行为建模与演化机制方面不断精细化,涌现出模仿机制、策略更新、羊群行为、异质偏好等多个研究路线,具备从微观规则生成宏观市场行为的能力。但目前主流 ASM 模型的外部扰动仍以噪声项或 exogenous signal 为主,缺少将社会网络作为“情绪渠道”的系统构造。例如,在\textcite{gao2020wealth} 的模型中,财富演化仅依赖于投资行为结果的反馈;而在\textcite{chen2021herding} 的模型中,协同行为更多依赖信号趋同性,而非社会交互结构。这限制了模型在解释“情绪如何形成”“恐慌如何蔓延”及“非理性行为如何传染”等方面的能力。

更值得注意的是,网络暴力作为一种在现实社交平台中频发的极端负面情绪传播现象,在学术研究中几乎未被建模。尽管已有部分研究意识到社交媒体中的悲观、愤怒、攻击性等情绪对市场的潜在风险,但仍缺乏对其传播机制、影响路径及调节机制的系统刻画。大多数研究将“情绪”简化为连续变量(如情感指数),并未区分攻击型、排斥型、消极型等复杂情绪的异质效应,更未考虑情绪扩散与个体行为反馈之间的互动关系。

此外,目前关于社交网络结构在金融市场中的作用研究仍以静态结构为主,缺乏对动态网络演化、结构异质性与行为反馈之间耦合关系的深入挖掘。尤其在面对具有传播性和攻击性的情绪(如网络暴力)时,社交图的拓扑结构、传播路径的异步性、被攻击者的响应策略(如沉默、反击)等因素将显著影响市场行为的系统性后果。

基于上述文献空白与不足,本文试图在以下三个方面作出拓展与创新:

\begin{itemize}
  \item 第一,将“网络暴力”作为独立研究对象引入金融行为建模框架,并区分其与一般社交情绪信号之间的传播机制差异,强调其攻击性、感染性和沉默效应;
  
  \item 第二,构建融合正反馈(被攻击者情绪极化转化为新攻击者)与负反馈(监管、举报、韧性增长)机制的社交传播模型,模拟网络暴力的动态演化过程;
  
  \item 第三,将该传播机制与Agent-Based人工市场系统耦合,构建情绪—行为—市场闭环反馈路径,通过仿真实验考察网络暴力在市场波动、财富分布与行为异质性演化中的作用。
\end{itemize}

综上,本文不仅致力于填补情绪金融与复杂系统建模之间的结合空白,也力图拓展人工股票市场模型的外部环境表达能力,在系统性地解释社交媒体如何通过网络暴力等极端舆情事件扰动金融市场方面提供理论支持与方法工具。
