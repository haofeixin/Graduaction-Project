\chapter{文献综述}

随着社交媒体平台和计算实验方法的发展,投资者行为的建模路径与金融市场的模拟研究逐步从“理性—均衡—封闭”模型,转向更贴近现实的“异质—互动—开放”框架。人工股票市场(Artificial Stock Market, ASM)与Agent-Based建模(ABM)方法日益融合,逐步成为行为金融与制度实验研究的重要工具。同时,社交媒体成为投资者观点传播、情绪共振与信息扩散的重要载体,社交网络中的极端行为(如网络暴力)逐渐引发金融学界关注。

本章将从两个方面系统梳理已有研究:一是人工股票市场与基于Agent的行为建模路径,二是社交媒体中的情绪传播与投资者行为机制。最后总结已有文献的研究空白并明确本文的定位。

\section{人工股票市场与基于Agent的建模研究}

人工股票市场建模强调投资者之间的行为差异、市场结构的显性建模与制度环境的机制模拟,是金融复杂系统建模的重要方向。\textcite{chiarella2009impact}构建了包含基本面投资者、技术交易者与噪声交易者三类Agent的市场模型,并通过订单簿交易机制模拟价格形成过程,成功复现了波动聚集、厚尾收益等市场特征,是现代ABM金融市场模型的奠基之作。

随后大量研究在此基础上发展出更具行为异质性与制度复杂性的模型框架。\textcite{zhang2021financialengineering}指出,ABM的可解释性、灵活性与机制透明性使其成为大数据驱动的金融实验平台。\textcite{li2012tick}基于连续双边报价系统,研究了最小报价单位变化对流动性的影响;\textcite{zhou2023shortselling}与\textcite{xiong2020shortsellingsize}则从融券卖空机制出发,模拟提价规则与规模约束对市场稳定性的作用,验证了市场制度在微观结构层面的调节功能。

行为建模方面,\textcite{hu2022heterogeneous}将前景理论与认知偏差引入投资者策略形成过程,发现风险偏好异质性与主观概率扭曲是市场剧烈波动的重要内生源。\textcite{chen2020herding}构建了羊群行为的协同扩散模型,说明个体模仿机制在推动集体非理性与市场异常中的作用;\textcite{zou2021chaotic}进一步指出,投资者反馈行为与策略演化可诱发市场的混沌波动。

近期研究趋势逐渐转向多层结构耦合与宏观系统稳定性评估。\textcite{zhangyi2022micromodel}设计了基于元模型的异质交易行为仿真框架,集成了交易者策略、市场机制与监管政策;\textcite{liangrui2022starboard}以科创板为案例,模拟交易机制改革对波动性与稳定性的双重影响。\textcite{wei2023t0}与\textcite{wei2021liquiditycrisis}分别研究了T+0交易制度与流动性踩踏机制,强调市场制度变化对系统风险演化路径的引导作用。

在国际研究方面,\textcite{gatti2020learningcovid}将COVID-19视作前所未有的外部冲击,通过ABM框架模拟投资者如何调整信念与行为,从而影响市场的结构性转变。\textcite{dawid2023behavioralheterogeneity}进一步将行为异质性视为金融市场宏观动态演化的核心变量,提出多群体间策略学习、社会互动与制度反馈的整合建模路径。

总体来看,ASM研究已具备较强的行为表达能力与制度实验基础,但在建模结构中尚缺乏投资者间社交传播、情绪传染与极端行为反馈的机制集成。

\section{社交媒体、投资者情绪与金融市场}

社交媒体在投资者情绪形成、信息扩散与行为反馈中扮演着愈发重要的角色。平台上的情绪表达不但反映市场预期,还可能通过群体模仿与社交传播引发集体行为偏差与市场异动。\textcite{cheng2020mediaemotion}基于微博数据构建了涨跌情绪指数,发现负面情绪与市场收益存在显著Granger因果关系,具有领先性预测能力。\textcite{shi2021attentionemotion}整合雪球、股吧与新闻评论构建情绪强度与关注度指标,指出社交关注是交易量与换手率波动的核心驱动因子。

\textcite{li2019emotionreturn}利用股吧评论文本提取投资者情绪指数,并结合Fama-MacBeth回归验证其对收益率的显著解释力,是社交数据情绪测度研究的重要代表。\textcite{liu2020abmreview}从理论角度指出,未来的金融行为建模应整合个体情绪状态、社交传播结构与市场微观机制,构建多层嵌套的耦合系统。

在建模路径上,\textcite{vanfossan2020abmsocial}首次将社交媒体情绪传播机制嵌入ABM市场结构中,构建了“市场—媒体—社交网络—投资者”多层交互路径,通过伪Agent设计模拟信息与情绪在不同媒介中的传播与反哺,开创了情绪传播与行为决策的耦合建模路径。

然而,已有文献多聚焦于情绪指数构建与传播路径模拟,较少涉及极端社交行为(如攻击性言论、群体压制)在市场系统中的反馈路径建模,缺乏对网络暴力这类情绪突发事件的行为干预与制度治理模拟。

\section{文献评述与本研究定位}

综上,已有文献在人工市场结构设计、异质性行为建模与社交媒体情绪量化等方面积累了丰富成果:

\begin{itemize}
  \item 在建模方法上,ASM结构已发展出多策略融合、制度调节与风险模拟等功能性机制;
  \item 在社交研究中,投资者情绪与市场行为之间的统计关系已得到充分验证;
  \item 部分研究开始探索社交媒体与金融市场之间的行为耦合机制,为社交传播模型与ABM融合奠定基础。
\end{itemize}

但在研究深度与机制集成方面仍存在明显空白:

\begin{itemize}
  \item 当前模型普遍缺乏对\textbf{网络暴力等极端负面情绪机制}的建模,无法解释攻击性情绪如何扰动投资者行为并反馈至市场系统;
  \item 多数研究未能构建\textbf{“社交传播—行为偏差—市场反馈”}的完整闭环,难以支持机制识别与政策实验;
  \item \textbf{情绪治理变量}(如举报、韧性成长、监管者介入)缺位,阻碍了模型在公共政策与行为干预方面的应用拓展。
\end{itemize}

本文拟构建一个集成网络暴力传播机制的多主体人工市场系统,在异质交易行为与连续双边市场结构基础上,引入社交网络结构、攻击传播模型与反馈调节机制,系统模拟“情绪—行为—市场”三层动态反馈路径,并评估情绪治理机制的调节效应,为金融系统中极端社交情绪的建模与治理提供理论工具。
