\chapter{导论}

\section{研究背景与意义}

随着社交媒体平台在公众生活中的日益普及,其在金融信息传播与投资者行为形成中的作用愈发重要。微博、股吧、雪球等平台已不再只是信息获取渠道,更成为投资者观点碰撞、情绪共鸣乃至群体决策的互动空间。社交媒体正在重塑市场主体的认知方式和行为路径,成为连接“信息流—情绪流—资金流”的关键载体。

近年来,学术界逐渐关注社交内容对金融市场的影响,尤其是社交情绪传播所引发的投资行为偏差与市场异动问题。已有研究表明,积极或消极情绪的集中释放往往会影响投资者的风险偏好与交易意愿,进而加剧价格波动、放大市场反馈效应,诱发“羊群行为”或“非理性繁荣”。

尽管相关文献在情绪构造与预测方面取得进展,但对于情绪在社交网络中如何扩散,并通过群体交互反馈影响市场的路径,仍缺乏系统刻画。特别是其中更极端的情绪表达形式——网络暴力(Cyberbullying)——作为攻击性、传染性并存的社会行为,尚未被有效纳入金融建模视野。

网络暴力的扩散具有非线性特征,常引发个体沉默、极端化或退场行为。这类行为不仅削弱市场多样性,也可能扰乱交易结构、降低定价效率,构成潜在的系统性风险。

另一方面,人工股票市场(Artificial Stock Market, ASM)作为行为金融与计算实验的重要工具,已广泛用于模拟投资者行为、制度变动与市场演化过程。ASM 可刻画异质性交易者在复杂环境下的交互行为,重现真实市场中的极端事件与财富集中等现象。然而,主流 ASM 模型往往未纳入社交机制,更缺乏对情绪传播与行为反馈链条的建模。

基于此,本文将网络暴力视为一种可传播、可反馈的社交情绪机制,嵌入 Agent-Based 人工市场系统中,系统探索其对投资者行为路径与市场效率的影响过程。该研究旨在弥补行为金融模型在“社交行为机制”刻画方面的空白,为理解社交平台风险如何传导至金融系统提供理论支撑与建模框架。

\section{研究内容与方法}

本文围绕网络暴力在社交网络中的传播机制及其对金融市场的影响路径展开,采用 Agent-Based 建模方法,构建集成“情绪传播—行为决策—市场反馈”机制的人工股票市场仿真系统。主要内容包括:

\begin{itemize}
  \item 构建具有攻击转化、情绪感染、沉默反应与治理反馈机制的网络暴力传播模型;
  \item 将传播机制嵌入人工市场系统,模拟网络暴力如何影响个体交易行为与市场状态;
  \item 设置对比实验,评估网络暴力机制对市场价格波动、流动性、财富分布等宏观结果的影响;
  \item 进一步分析不同机制参数(如监管强度、心理韧性、社交结构)对系统演化的调节作用。
\end{itemize}

研究方法综合使用了多主体行为建模、社交网络传播建模、连续双边报价市场设计、蒙特卡洛仿真、配对 T 检验与可视化分析,力求在行为层与市场层构建完整耦合机制。

\section{研究创新与不足}

本文在研究问题、模型结构与机制设计方面具有如下创新:

\begin{itemize}
  \item \textbf{问题视角新颖}:首次将网络暴力作为可传播、可反馈的社会行为机制引入人工市场模拟,关注其对个体行为与市场效率的双重影响;
  \item \textbf{机制设计复杂}:构建包含正反馈(攻击转化)与负反馈(举报、韧性恢复)的传播机制,刻画网络暴力在社交网络中的动态演化过程;
  \item \textbf{结构耦合完整}:集成“情绪传播—行为决策—市场反馈”链条,形成“情绪—行为—市场”的反馈闭环建模框架;
  \item \textbf{评估维度多元}:从个体层(情绪偏差、交易行为)、群体层(财富分布、行为极化)和系统层(流动性、信息效率)多维度开展分析。
\end{itemize}

同时,本文也存在若干局限性:

\begin{itemize}
  \item 模型参数主要基于理论设定,缺乏真实社交或市场数据校准;
  \item 网络结构为静态预设,未考虑社交连接随行为动态调整;
  \item 网络暴力情绪建模较为抽象,未纳入文本情绪、语义攻击强度等精细表达。
\end{itemize}

上述不足为后续模型拓展与实证研究提供了方向。

\section{章节安排}

本论文共分为五章,结构如下:

\begin{itemize}
  \item 第一章为导论,介绍研究背景、研究内容、创新点与论文结构;
  \item 第二章为文献综述,系统梳理人工市场建模、社交网络传播与情绪金融相关研究;
  \item 第三章为模型设定,介绍投资者行为模型、市场结构与网络暴力传播机制;
  \item 第四章为实验设计与仿真分析,对比有无网络暴力机制下系统演化特征;
  \item 第五章为结论与展望,总结主要发现,反思模型局限,并提出后续研究方向。
\end{itemize}
