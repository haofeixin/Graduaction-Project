\chapter{导论}

\section{研究背景与意义}

随着社交媒体平台在公众生活中日益普及,其在金融信息传播与投资者行为形成中的作用愈发凸显。投资者不仅通过微博、股吧、雪球等平台获取市场资讯、交流观点,还通过发帖、评论、转发等方式参与情绪共鸣与意见塑造,社交媒体已成为连接信息流、情绪流与资金流的关键载体。

在此背景下,学术界逐渐关注社交媒体内容对金融市场的影响,尤其是投资者情绪在平台上传播所引发的群体行为偏差、价格异动与市场波动问题。已有研究指出,积极或消极情绪的集中爆发会影响投资者的交易意愿与风险偏好,从而导致价格脱离基本面,形成“羊群效应”或“非理性繁荣”等市场异常现象。

然而,大部分研究聚焦于社交情绪的构造与预测能力,对情绪如何在投资者群体中传播并反馈至市场缺乏机制性刻画。特别是其中的极端负面情绪——如网络暴力(Cyberbullying)——尚未被系统纳入金融建模范式。网络暴力不仅具有攻击性和传染性,还可能引发沉默、焦虑、极端化等行为反应,其在社交网络中呈现出的非线性扩散特征可能会引发集体非理性行为,对市场稳定性构成潜在威胁。

另一方面,人工股票市场(Artificial Stock Market, ASM)作为行为金融与计算实验的重要工具,在解释市场波动、模拟投资行为和制度测试方面已取得显著进展。ASM可灵活模拟异质性Agent的交易行为,并结合复杂系统理论重现真实市场中的非对称波动、极端事件与财富演化。然而,目前主流ASM研究仍未考虑社交网络情绪传播机制,特别是网络暴力等攻击性行为在市场系统中的影响路径。

基于此,本文拟将网络暴力作为一种可传播、可反馈的社会行为机制引入人工股票市场框架,探究其如何通过社交网络影响投资者行为决策,并进一步扰动市场价格与财富分布结构。该研究不仅有助于拓展情绪金融的理论视野,也为理解社交媒体风险在金融市场中的传导机制提供建模依据。

\section{研究内容与方法}

本文以社交媒体中的网络暴力传播为研究对象,结合人工股票市场模型与Agent-Based建模方法,构建融合“情绪传播—行为反馈—市场响应”的仿真系统,主要内容包括:

\begin{itemize}
  \item 设计具有攻击性和反馈机制的网络暴力传播模型,刻画情绪感染、沉默反应、攻击转化等关键行为;
  \item 将传播机制嵌入ABM金融市场系统,模拟情绪状态如何影响个体交易行为;
  \item 通过仿真实验对比有无网络暴力情境下市场指标(如波动率、流动性、财富集中度)变化;
  \item 评估不同机制参数(如监管强度、韧性成长、举报概率)对系统演化结果的调节作用。
\end{itemize}

研究方法主要包括:多主体行为建模、社交网络传播建模、订单簿市场结构设计、Monte Carlo仿真、配对T检验、图形可视化等。

\section{研究创新与不足}

本文在研究视角与模型设计上具有如下创新点:

\begin{itemize}
  \item \textbf{研究问题新颖}:区别于以往研究社交媒体舆情或情绪指数对市场的静态影响,本文将“网络暴力”作为一种具有传播性、行为诱导性的社会机制纳入建模;
  \item \textbf{传播机制设计复杂}:构建包含正反馈(攻击者转化)、负反馈(举报、监管、韧性成长)的社交传播模型,模拟现实社交环境中的复杂情绪动态;
  \item \textbf{建模路径融合}:首次在ABM金融市场中集成“社交传播—交易行为—市场结构”完整链条,形成“情绪-行为-市场”反馈闭环;
  \item \textbf{评估体系全面}:从个体层面(交易活跃度、情绪偏差)、群体层面(财富分布、行为极化)和市场层面(流动性、价格波动)开展系统分析。
\end{itemize}

同时,本文也存在一些局限:

\begin{itemize}
  \item 模型假设与参数设定依赖于理论与启发式判断,缺乏真实数据校准;
  \item 社交网络结构为预设静态图,未考虑动态演化或结构自适应;
  \item 网络暴力定义为统一攻击性变量,未区分具体语言、行为形态等多维细节。
\end{itemize}

\section{章节安排}

本论文共分为六章,结构如下:

\begin{itemize}
  \item 第一章为导论,介绍研究背景、研究内容、创新点与章节安排;
  \item 第二章为文献综述,系统整理人工市场与社交媒体影响的相关研究;
  \item 第三章为模型构建,介绍投资者行为模型、网络暴力传播机制与市场结构;
  \item 第四章为实验设计与结果分析,描述仿真流程与市场行为数据对比;
  \item 第五章为扩展分析,探讨多参数下的系统表现与机制敏感性;
  \item 第六章为结论与展望,总结研究结论,反思不足,并提出后续研究方向。
\end{itemize}
