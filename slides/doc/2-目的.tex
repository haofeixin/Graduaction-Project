\begin{frame}
    \begin{center}
        \textcolor{NJU_purple}{\Large 第二部分} \\[1.5em]
        \textcolor{NJU_purple}{\Huge 研究目的与创新点}
    \end{center}
\end{frame}



\begin{frame}{研究目的与创新点}

    
    \justifying
    \footnotesize
    \setlength{\parskip}{0.3em}
    
    
    \textbf{\large 研究目的:}
    
    \vspace{0.2cm} % 研究目的标题和正文间距
    
    本研究旨在构建一个集成网络暴力传播机制的人工股票市场模型,系统探讨网络暴力如何通过投资者情绪传播影响散户行为、市场效率及财富分布,
    \alert{\textbf{为监管和情绪风险预警提供理论依据。}}
    
    % 下面创新点照旧
    \vspace{0.5cm}
    
    \textbf{\large 主要创新点:}
    \begin{itemize}
      \item \alert{首次引入网络暴力(Cyberbullying)概念},填补行为金融和市场微观结构模型中缺乏该研究的空白。引入\alert{正负反馈机制},模拟真实世界中网暴的传导与抑制。
      
      \item 基于经典文献\textbf{Chiarella et al. (2009)},在其订单驱动市场模型基础上进行多项创新改进:\\
      \hspace{1em} - 交易者执行模式新增三种(单一、部分、多重同时交易),更符合现实市场动态;\\
      \hspace{1em} - 参数分布由指数改为\alert{对数正态分布},更符合投资者异质性特征;\\
      \hspace{1em} - 在散户决策模型中引入\alert{情绪偏差 (emotion\_bias)},模拟网络暴力影响下的投资行为变化。
      
      \item 细化机构与散户投资者行为,重点关注网络暴力对\alert{散户群体财富和行为}的影响,紧扣国家资本市场“加强投资者获得感”政策目标。
      
      \item 使用\alert{LaTeX和Beamer完成论文及PPT制作},结合
      \href{https://github.com/haofeixin/Graduaction-Project}{\textcolor{blue}{GitHub代码仓库}}
      进行版本管理,保障研究成果规范性与可复现性。
    \end{itemize}
    
    \end{frame}
    