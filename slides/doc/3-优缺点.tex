
\begin{frame}{目录}
        \begin{center}
            \textcolor{NJU_purple}{\Large 第三部分} \\
            \text{\;} \\
            \textcolor{NJU_purple}{\Huge GRU的应用} 
        \end{center}      
    \end{frame}

\begin{frame}{GRU的应用场景}
    \begin{itemize}
        \item 自然语言处理(NLP)
        \begin{itemize}
            \item 机器翻译:捕捉源语言和目标语言的上下文依赖(如Google的早期翻译模型)。
            \item 文本生成:生成连贯的对话或文章(如聊天机器人、自动摘要)。
            \item 情感分析:分析长文本中的情感倾向。
        \end{itemize}
        \item 时间序列预测
        \begin{itemize}
            \item 股票价格预测:基于历史价格序列预测未来趋势。
            \item 气象预测:处理时间相关的气象数据(如温度、湿度序列)。
        \end{itemize}
        \item 语音识别
        \begin{itemize}
            \item 声学建模:将语音信号映射为文本序列,捕捉语音中的时序特征。
        \end{itemize}
        \item 推荐系统
        \begin{itemize}
            \item 用户行为建模:根据用户历史行为序列(点击、购买记录)预测兴趣。
        \end{itemize}
    \end{itemize}
\end{frame}


\begin{frame}{GRU的优缺点}
    \begin{itemize}
        \item 优势:
        \begin{itemize}
            \item 简洁高效:GRU的结构相对简单,参数较少,训练速度快。
            \item 解决梯度问题:通过引入门机制,GRU有效地解决了传统RNN中的梯度消失和爆炸问题,从而能够更好地捕捉序列数据中的长期依赖关系。
            \item 适应性强:可以用于处理各种类型的序列数据,包括文本、音频、图像等。
        \end{itemize}
        \item 限制:
        \begin{itemize}
            \item 对于非常长的序列,GRU可能无法完全捕捉所有的长期依赖关系。因为尽管门机制帮助控制信息的传递,但在非常长的序列中信息的传递仍会受到一定的限制。
            \item GRU难以显式建模序列中的层次结构。如,在自然语言处理任务中,词语的含义可能取决于它在句子中的位置,而句子的含义可能取决于它在段落中的位置。这种层次结构是GRU难以处理的。
        \end{itemize}
    \end{itemize}
\end{frame}


\begin{frame}{LSTM VS GRU}
\begin{table}[!htbp]
	\centering
    \resizebox{1.0\textwidth}{!}{
	\begin{tabular}{|c|c|c|}
		\hline
		对比维度   & LSTM               & GRU                \\ \hline
		门控机制   & 3个门:输入门、遗忘门、输出门    & 2个门:更新门、重置门        \\ \hline
		记忆单元   & 独立细胞状态(Cell State) & 无独立细胞状态,通过更新门和重置门联合控制    \\ \hline
		参数量    & 较多(多一个门控和细胞状态)     & 较少(参数更精简)          \\ \hline
		计算复杂度  & 较高(需维护细胞状态)        & 较低(合并门控和状态)        \\ \hline
		训练速度   & 较慢(参数多)            & 较快(参数少)            \\ \hline
		长依赖捕捉  & 更强(显式控制记忆遗忘)       & 稍弱(隐式记忆更新)         \\ \hline
		适用场景   & 超长序列、复杂时序依赖(如机器翻译) & 中等序列、实时性要求高(如语音识别) \\ \hline
		梯度消失问题 & 缓解(通过细胞状态)         & 缓解(通过更新门)          \\ \hline
		主流框架实现 & 广泛支持               & 广泛支持               \\ \hline
	\end{tabular}}
	\caption{GRU和LSTM对比}
	\label{tab:my-table}
\end{table}
\end{frame}

