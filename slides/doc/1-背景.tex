\begin{frame}
    \begin{center}
        \textcolor{NJU_purple}{\Large 第一部分} \\[1.5em]
        \textcolor{NJU_purple}{\Huge 研究背景与意义}
    \end{center}
\end{frame}


\begin{frame}{研究背景与意义}

    \setlength{\parskip}{0.2em} % 段落间距调整
    \footnotesize % 适合答辩页的字号
    
    \setbeamercolor{block title}{fg=white,bg=blue!80}
    \setbeamercolor{block body}{fg=black,bg=blue!15}
    \begin{block}{社交媒体与投资者行为}
    随着社交媒体平台(如微博、雪球等)在公众生活中的快速普及,投资者获取信息和表达情绪的渠道日益多样化和便捷化。社交媒体不仅作为信息传播载体,更成为投资者观点碰撞、情绪共鸣及群体决策的重要场所。
    \alert{\textbf{社交媒体已成为连接“信息流—情绪流—资金流”的关键枢纽}},其情绪传播机制对投资者的风险偏好、交易决策及市场行为产生深远影响,甚至可能引发非理性繁荣和市场异动。
    \end{block}
    
    \vspace{0.1cm}
    
    \setbeamercolor{block title}{fg=white,bg=red!80}
    \setbeamercolor{block body}{fg=black,bg=red!15}
    \begin{block}{网络暴力的挑战与研究空白}
    在社交媒体平台上,部分投资者间的情绪表达可能演变为\alert{\textbf{网络暴力(Cyberbullying)}},表现为针对少数异见者的攻击性言论和行为,具有高度传染性和破坏性。这种现象可能导致少数意见群体的失声(沉默),引发市场信息有效性的丧失及意见极化(回声室效应),最终削弱市场定价效率并影响散户财富分布。
    然而,现有金融市场模型普遍缺乏对网络暴力等极端情绪传播机制的关注,尚未构建\alert{\textbf{“情绪传播—行为偏差—市场反馈”}}的完整闭环模型,难以有效揭示其对市场效率和财富分配的具体影响。
    \end{block}
    
    \vspace{0.1cm}
    
    \setbeamercolor{block title}{fg=black,bg=gray!20}
    \setbeamercolor{block body}{fg=black,bg=gray!7}
    \begin{block}{本研究意义}
    本研究基于\alert{\textbf{Agent-Based建模方法}},设计并实现了集成网络暴力传播机制的人工股票市场仿真系统,系统性探讨网络暴力如何通过情绪-行为反馈影响投资者的交易意愿和市场效率。
    研究成果不仅丰富了行为金融与市场微观结构的理论体系,也为金融监管部门和社交平台提供了\alert{\textbf{情绪监管与市场风险预警}}的理论支持和模型基础。
    \end{block}
    
    \end{frame}
    