
    \begin{frame}{字体}
        \justifying
        
        \begin{columns}
            \column{0.3\textwidth}
            \centering
            \begin{itemize}
                \item 默认字体
                \item \songti{宋体字体}
                \item \heiti{黑体字体}
                \item \kaishu{楷体字体}
            \end{itemize}
            \begin{enumerate}
                \item default font
                \item \textrm{serif font}
                \item \textsf{sans serif font}
                \item \texttt{mono font}
            \end{enumerate}  
            \column{0.3\textwidth}
            \begin{itemize}
                \item 字体格式
                \begin{itemize}
                    \item \textbf{bold}
                    \item \textit{italic}
                    \item \alert{alert}
                    \item \emph{emphasize}
                \end{itemize}
            \end{itemize}
            \column{0.4\textwidth}
            \begin{itemize}
                \item 一级无序列表
                \begin{itemize}
                    \item 二级无序列表
                    \begin{itemize}
                        \item 三级无序列表
                    \end{itemize}
                \end{itemize}
            \end{itemize}
            \begin{enumerate}
                \item 一级有序列表
                \begin{enumerate}
                    \item 二级有序列表
                    \begin{enumerate}
                        \item 三级有序列表
                    \end{enumerate}
                \end{enumerate}
            \end{enumerate}
        \end{columns}
          
    \end{frame}

    \begin{frame}{块(非数学)} 
    
        \begin{block}{普通块}
            block content
        \end{block}
        \begin{alertblock}{高亮块}
            alertblock content
        \end{alertblock}
        \begin{exampleblock}{案例}
            exampleblock (case) content
        \end{exampleblock}
        \begin{example}
            example content
        \end{example}
        
    \end{frame}

    \begin{frame}{块(数学)} 
    
        \begin{definition}
            definition content
        \end{definition}
        \begin{proof}
            proof  content
        \end{proof}
        \begin{theorem}
            theorem content
        \end{theorem}
        \begin{lemma}
            lemma content
        \end{lemma}
        \begin{corollary}
            corollary content
        \end{corollary}
        
    \end{frame}